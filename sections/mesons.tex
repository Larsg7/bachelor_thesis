\subsection{Meson correlation function}
    Mesons are quark-antiqaurk states which can be characterized by the following QCD quantum numbers \cite{introduction_to_lattice_hadron_spectroscopy}:
    \begin{itemize}
        \item Spin and total angular momentum $J$, $J = 0, 1, 2,...$ (all mesons are bosons)
        \item Parity $P = \pm1$
        \item Charge conjugation $C = \pm1$
        \item Flavour quantum numbers:
            \begin{itemize}
                \item $u$ and $d$: Isospin $I$; $I_z = \pm\frac{1}{2}$
                \item $s$ and $\bar{s}$: Strangeness $S$; $S = \pm 1$
                \item $c$ and $\bar{c}$: Charmness $C$; $C = \pm 1$
                \item $b$ and $\bar{b}$: Bottomness $B'$; $B' = \pm 1$
                \item $t$ and $\bar{t}$: Topness $T$; $T = \pm 1$
            \end{itemize}
        \item Electric charge is neglected here.
    \end{itemize}
    
    \subsubsection{Creation operators}
        If one wants to examine a meson with a particular set of quantum numbers $I(J^P)$, $I$ denotes the total isospin, the correspoding creation operator $\Operator(n)$ needs to be defined. When acting on the vacuum $\Operator(n)$ creates a so called trial state $|\phi,n\rangle$ with the quantum numbers $I(J^P)$:
        \begin{equation}
            \Operator(n)|\Omega\rangle = |\phi,n\rangle
        \end{equation}
        Meson creation operators can be assembled in different ways, the easiest is
        \begin{equation}\label{meson_creation_operator}
            \begin{aligned}
                \Operator(n) &= \bar{\psi}^{(f_1)}(n)\Gamma\psi^{(f_2)}(n)\\
                \Operator^\dagger(n) &= \pm\bar{\psi}^{(f_2)}(n)\Gamma\psi^{(f_1)}(n)
            \end{aligned}
        \end{equation}
        $\Gamma$ is a matrix in Dirac space which carries total momentum $J$ and parity $P$. The sign in \mref{meson_creation_operator} depends on the nature of $\Gamma$, because $\gamma_4\Gamma^\dagger\gamma_4 = \pm \Gamma$. The following table lists the possible configurations for $\Gamma$:
        \begin{table}[h]
            \centering
            \begin{tabular}{|c|c|c|c|}
            \hline
            \multicolumn{1}{|c|}{$\Gamma$} & \multicolumn{1}{c|}{$J$} & \multicolumn{1}{c|}{$P$} & \multicolumn{1}{c|}{Transformation} \\ \hline
             1,$\gamma_4$ & 0 & +1 &    scalar           \\
             $\gamma_5, \gamma_4\gamma_5$                      &  0                     &  -1                     &  pseudoscalar                     \\
             $\gamma_i,\gamma_4\gamma_i$                      & 1                      &    -1                   &  vector                     \\
             $\gamma_4\gamma_i$                      & 1                      &  +1                     & axial vector                      \\
              $\gamma_i\gamma_j$                     & 1                      & +1                      & tensor                     
            \end{tabular}
            \caption{Possible options for $\Gamma$}
            \label{my-label}
        \end{table}
    \subsubsection{Meson correlation functions}
        To determine the energy of an eigenstate of the hamiltonian of a quantum field theory one can calculate the vacuum expectation value (VEV) of the respective creation operator. Also called the correlator, the VEV is defined by the Feynman path integral:
        \begin{equation}\label{VEV}
            \begin{aligned}
                C(t_2 - t_1) &\equiv \langle\Omega|\Operator^\dagger(t_2)\Operator(t_1)|\Omega\rangle \\
                &= \frac{1}{Z}\int D[\psi]D[\bar{\psi}]D[A]\ \Operator^\dagger(t_2)\Operator(t_1)e^{-S[\psi, \bar{\psi}, A]}
            \end{aligned}
        \end{equation}
        where $Z$ defines the partition function:
        \begin{equation}\label{partition_function}
            Z = \int D[\psi]D[\bar{\psi}]D[A]\ e^{-S[\psi, \bar{\psi}, A]}
        \end{equation}
        $S[\psi, \bar{\psi}, A]$ is the lattice action. Introducing a finite lattice spacing difficulties calculating the path integral are solved. The integral becomes regularized and the infinite dimensional integral can be replaced by a finite product of finite dimensional integrals:
        \begin{equation}
            \int D[\psi]D[\bar{\psi}]D[A]\ \approx \prod_{x \in \Lambda}\int d\psi(x)d\bar{\psi(x)}dA(x)
        \end{equation}
        To achieve the goal of this work, calculating the mesinic energy spectrum, one introduces a complete set of energy eigenstates $\sum_k |k\rangle\langle k| = 1$ into equation \mref{VEV}:
        \begin{equation}\label{correlation_with_eigenstates}
            C(t_2 - t_1) = \sum_k \langle\Omega|\Operator^\dagger(t_2) |k\rangle\langle k| \Operator(t_1)|\Omega\rangle
        \end{equation}
        We can now exploit that $\Operator^\dagger(t_2)$ is a Heisenberg operator and its temporal evolution can be described by
        \begin{equation}
            \Operator^\dagger(t_2) = e^{H(t_2 - t_1)}\Operator^\dagger(t_1)e^{-H(t_2 - t_1)}
        \end{equation}
        So we can write \mref{correlation_with_eigenstates} as
        \begin{equation}
            \begin{aligned}
                C(t_2 - t_1) &= \sum_k \langle\Omega|e^{H(t_2 - t_1)}\Operator^\dagger(t_1)e^{-H(t_2 - t_1)} |k\rangle\langle k| \Operator(t_1)|\Omega\rangle\\
                &= \sum_k \langle\Omega|e^{E_\Omega(t_2 - t_1)}\Operator^\dagger(t_1)e^{-E_k(t_2 - t_1)} |k\rangle\langle k| \Operator(t_1)|\Omega\rangle\\
                &= \sum_k (\langle k|\Operator(t_1) |\Omega\rangle)^\dagger\langle k| \Operator(t_1)|\Omega\rangle e^{(E_\Omega - E_k)(t_2 - t_1)}\\
                &= \sum_k |\langle k|\Operator(t_1) |\Omega\rangle)|^2 e^{E(t_2 - t_1)}\\
            \end{aligned}
        \end{equation}
        where $E = E_\Omega - E_k$ is the energy of the energystate $|k\rangle$ relative to the vacuum energy.