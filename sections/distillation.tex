	
% descibe distillation
The term distillation describes a quark-field smearing algorithm to compute hadron correlation functions. It was first described in 2009 by Michael Peardon \cite{distillation_paper}. Distillation promises efficient computations of all-to-all quark propagators.\\

\subsection{Distillation operator}
    Smearing is a method to ??? used to apply a smoothing function to the field before applying the creation operators. Smearing should remove short-range modes while keeping as many symmetries as possible. Distillation makes use of a gauge-covariant quark smearing based on the lattice Laplacian which is defined as follows
	\begin{equation}
	    -\nabla^2_{xy} = 6\delta_{xy} - \sum^3_{j=1}(\tilde{U}_j(x,t)\delta_{x+j,y} + \tilde{U}^\dagger_j(x-y,t)\delta_{x-j,y})
	\end{equation}
	
	Starting from the lattice Laplacian one can define a simple smearing operator
	\begin{equation}
	    J_{\omega,n}(t) = (1+\frac{\omega\nabla^2(t)}{n})^n
	\end{equation}
	
	$\omega$ and $n$ are tunable parameters. For large $n$, $J$ defines the exponential of $\omega\nabla^2$ which suppresses higher eigenmodes of the laplacian hence only a few of the lowest modes contribute to $J$. % WHY?
	\begin{equation}
        \lim_{n\rightarrow\infty} J_{\omega,n}(t) = \exp(\omega\nabla^2) \equiv J_\omega
	\end{equation}
	
    The lattice laplacian is a gauge-covariant, linear, negative-definite and hermitian operator acting on a $M$ dimensional Hilbert space. On each timeslice its eigenvectors are orthogonal and one can find an orthonormal basis of $\mathbb{C}^M$
    \cite{bachelor_thesis_jan}
    \begin{equation}\label{othonormal_eigenvectors}
        \sum^M_{k=1}\ket{v_k}\bra{v_k} = 1
    \end{equation}
    where $v_k$ defines the $k^{th}$ eigenvector on the corresponding timeslice.
    \begin{equation}
        \nabla^2\ket{v_k} = \lambda_k\ket{v_k}
    \end{equation}
    The eigenvalues of the lattice laplacian are semi-negative $\lambda_k \in (-\infty,0]$ and $\lambda_{k+1} < \lambda_k$.\\
    
    Using the above we can now expand $J_\omega$ in the limit $n\rightarrow\infty$ using \mref{othonormal_eigenvectors}:
    \begin{equation}
        \begin{aligned}
            J_\omega &= \sum^M_{k=1}\ket{v_k}\bra{v_k} J_\omega\\
            &= \sum^M_{k=1}\ket{v_k}\bra{v_k} \exp(\omega\nabla^2)\\
            &= \sum^M_{k=1}\ket{v_k}\bra{v_k} \exp(\omega\lambda_k)
        \end{aligned}
    \end{equation}
    One can see that higher eigenmodes are suppressed exponentially. Therefore there is some number $N<<M$ such that $\exp(\omega\lambda_k) << 1$ holds. So we can write
    \begin{equation}
        J_\omega \approx \sum^N_{k=1}\ket{v_k}\bra{v_k} \exp(\omega\lambda_k)
    \end{equation}
    This motivates the definition of the distillation operator which is constructed by \cite{distillation_paper}:
    \begin{equation}
        \Box_t \equiv V_tV_t^\dagger
    \end{equation}
    $V_t$ is a $M \times N$ matrix which $k^{th}$ column contains the $k^{th}$ eigenvector of $\nabla^2$ evaluated on the $t^{th}$ timeslice, sorted by eigenvalue. $\Box_t$ can also be written in terms of the eigenvectors
    \begin{equation}
        \Box_{t}(\textbf{x},\textbf{y}) = \sum^N_{k=1}v_{k,t}(\textbf{x})v_{k,t}(\textbf{y})
    \end{equation}
    $\Box$ projects into $V_N$, the subspace spanned by the $N$ lowest eigenmodes, hence $\Box^2 = \Box$. Quark fields can be smeared by applying the distillation operator onto them. They inherit all symmetry properties of the unsmeared fields.
    \begin{equation}
        \chi^{(f)}(\textbf{x},t) \equiv \sum_{\textbf{y}} \Box_t(\textbf{x},\textbf{y}) \psi^{(f)}(\textbf{y},t)
    \end{equation}
    
\subsection{Distilled meson two-point correlation functions}
    Meson two-point correlation functions of disilled fields can be constructed similar to \mref{VEV}:
    \begin{equation}
        \begin{aligned}
            C(t_2 - t_1) &= \langle\Omega| \Operator^\dagger(t_2) \Operator(t_1)|\Omega\rangle\\
            = \pm\langle&\Omega| \bar{\psi}^{(f_1)}(\textbf{x}_1,t_2)\Box_{t_2}(\textbf{x}_1,\textbf{y}_1)\Gamma^A
            \Box_{t_2}(\textbf{y}_1,\textbf{x}_2)\psi^{(f_1)}(\textbf{x}_2,t_2)\\
            &\times \bar{\psi}^{(f_2)}(\textbf{x}_3,t_1)\Box_{t_1}(\textbf{x}_3,\textbf{y}_2)\Gamma^B
            \Box_{t_1}(\textbf{y}_2,\textbf{x}_4)\psi^{(f_2)}(\textbf{x}_4,t_1) |\Omega\rangle
        \end{aligned}
    \end{equation}

    Executing the same steps as in \mref{correlator_final} yields
    \begin{equation}
        \begin{aligned}
            C(t,t') &= \langle Tr[\Phi^B(t')\tau(t',t)\Phi^A(t)\tau(t,t')] \rangle
        \end{aligned}
    \end{equation}
    In the last line the following definitions were used
    \begin{equation}
        \begin{aligned}\label{def_gamma_and_perambulator}
            \Phi_{AB}^A(t) &\equiv V^\dagger(t)\Gamma_{AB}^A(t)V(t)\\
            \tau_{AB}(t,t') &\equiv V^\dagger(t)(D^{-1})_{AB}(t,t')V(t')
        \end{aligned}
    \end{equation}
    $\Phi^A(t)$ is called the distilled gamma matrix and $\tau(t,t')$ the perambulator. The latter contains information about the quark propagator and can be independently computed from the former.
    
\subsection{Computation of the perambulator}
    The propagator $(D^{-1(f)})^{ab}_{AB}(x;y)$ is defined by the following equation \cite{four_quark_correlation_functions}
    \begin{equation}
        \sum_{y}(D^{(f)})^{ab}_{AB}(x;y)(D^{-1(f)})^{bc}_{BC}(y;z) = \delta_{ac}\delta_{AC}\delta(x,z)
    \end{equation}
    To calculate the perambulator one computes single inversions $\psi(x)$ defined by the linear equation
    \begin{equation}
        \begin{aligned}\label{inversion_term}
            &\sum_{x}(D^{(f)})^{ab}_{AB}(x;y)\psi^{(f)b}_{B}(y) = \xi^a_A(x)\\
            &\psi^{(f)b}_{B}(y) = \sum_{x}(D^{-1(f)})^{ba}_{BA}(y,x)\xi^a_A(x)
        \end{aligned}
    \end{equation}
    where $\xi(x)$ is the so called source term, also a fermionic field. Recall the definition of the perumbulator \mref{def_gamma_and_perambulator}, writing it explicitly will enable us to identify the inversion in the last equation \cite{bachelor_thesis_jan}
    \begin{equation}\label{perambulator_explicit}
        \begin{aligned}
            &\tau^{(f)}_{AB}(t,t') \equiv V^\dagger_t(D^{-1})_{AB}(t,t')V_{t'}\\
            &= \sum_{k=1,k'=1}^N \sum_{x,y}
            v_{k,t}^{\dagger a}(\textbf{x}) 
            (D^{-1(f)})^{ab}_{AB}(\textbf{x},t;\textbf{y},t')
            v_{k',t'}^{b}(\textbf{y})\\
            &= \sum_{k=1,k'=1}^N \sum_{\textbf{x},\textbf{y},x_0,y_0}
            v_{k,t}^{\dagger a}(\textbf{x})\delta_{AA'}\delta_{tx_0}
            (D^{-1(f)})^{ab}_{A'B'}(\textbf{x},x_0;\textbf{y},y_0)
            v_{k',t'}^{b}(\textbf{y})\delta_{BB'}\delta_{t'y_0}
        \end{aligned}
    \end{equation}
    The inversion \mref{inversion_term} can now be identified
    \begin{equation}
        \psi^{(f)a}_{A;B,t',k'}(\textbf{x},x_0) = (D^{-1(f)})^{ab}_{A'B'}(\textbf{x},x_0;\textbf{y},y_0)
            v_{k',t'}^{b}(\textbf{y})\delta_{BB'}\delta_{t'y_0}
    \end{equation}
    The indices $B,t',k'$ mean that $\psi^{(f)a}_{A;B,t',k}(\textbf{x},t)$ is defined on time slice $t'$ and spin index $B$ for the $k'^{th}$ eigenvector.\\
    
    The source terms $\xi(x)$ can also be identified:
    \begin{equation}
        \xi^a_{B;B',t',k'}(\textbf{y},y_0) =  v_{k',t'}^{b}(\textbf{y})\delta_{BB'}\delta_{t'y_0}
    \end{equation}
    The source terms are fields of the needed size $L^3 \times 4 \times 3$ but have non-zero entries only at one specific spin index and timeslice.\\
    
    Finally, equation \mref{perambulator_explicit} becomes:
    \begin{equation}\label{perambulator_with_inversion}
        \tau^{(f)}_{A'B'}(t,t') = \sum_{k=1,k'=1}^N \sum_{\textbf{x},x_0} \xi^{\dagger a}_{A;A',t,k}(\textbf{x},x_0) \psi^{(f)a}_{A;B',t',k'}(\textbf{x},x_0) 
    \end{equation}
    
    To compute the complete perambulator one needs to compute $T \times 4 \times N$ inversions for all values of $B,t',k$ and then multiply these with the hermitian conjugate of the sources as shown in equation \mref{perambulator_explicit}. The eigenvectors were computed using code by ????, the source terms were calculated using code written by \textit{Jan Kruse} \cite{bachelor_thesis_jan} and then submitted to the program \verb+tmLQCD+ \cite{jansen_urbach_2009} which calculates the inversions. Details about the implementation can be found in section \ref{section:implementation}.
    
\subsection{Problems of distillation}
    