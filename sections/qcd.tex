Throughout this work natural units will be used, so that $\hbar = c = a = 1$, where $a$ is the distance between two lattice points.	In any instance where one index appears twice the \textbf{Einstein summation notation} was used, an implicit summation is assumed. When working with path integrals the euclidean formulation of QCD is used where the real time $x_0$ is replaced with the imaginary time $x_0 = -ix_4$. Therefore the Minkowsky tensor $g_{\mu\nu}$ is replaced by $\delta_{\mu\nu}$.
	
\subsection{QCD}
	Quantum chromodynamics is a SU(3) gauge quantum field theory and describes quarks and gluons. The QCD Lagrangian is defined as \cite{qcd1_script_philipsen}
	\begin{equation}
	    \Lagr_{QCD}[\psi,\bar{\psi}, A] = \sum_f \bar{\psi}_f(x)(i\gamma_\mu\GaugeDerivative_\mu + m_f)\psi_f(x) + \frac{1}{4}F^a_{\mu\nu}F^a_{\mu\nu}
	    \label{qcd_lagrangian}
	\end{equation}
	$\psi(x)$ represents the quark field, $\GaugeDerivative_\mu = \partial_\mu - igA_\mu(x) = \partial_\mu-igT^aA^a_\mu(x)$ is the gauge covariant derivative and $F_{\mu\nu} = T^aF^a_{\mu\nu}$ is the field strength tensor of the theory. The latter is defined as
	\begin{equation}\label{strenth_tensor}
	\begin{aligned}
	    F_{\mu\nu} &= \partial_\mu A_\nu - \partial_\nu A_\mu-ig[A_\mu,A_\nu]\\
	    &= T^a(\partial_\mu A^a_\nu - \partial_\nu A^a_\mu + g f^{abc}A^b_\mu A^c_\nu)
	\end{aligned}
	\end{equation}
	\noindent
	The $T^a$'s are called $SU(3)$ generators and are part of the $SU(3)$ Lie algebra with the commutator relationship $[T^a,T^b]=if^{abc}T^c$. The action is given by
	\begin{equation}
	    S_{QCD}[\psi,\bar{\psi}, A] = \int d^4x \Lagr_{QCD}[\psi,\bar{\psi}, A]
	\end{equation}
	and is invariant under local $SU(3)$ transformations. It is also invariant under $U(1)$ transformations. The first term of the lagrangian and hence the action displays information about the fermionic dynamic of the theory, the second term about the self interaction of the gluons.
	
	The internal degrees of freedom of the quark fields $\psi$ and $\bar{\psi}=\psi^\dagger\gamma_4$ separate into three parts: Spin ($A$), color ($c$) and flavor ($f$) with 4, 3, and $N_f$ components respectively. The gluon fields have eight color degrees of freedom and transform under the adjoint representation of $SU(3)$.