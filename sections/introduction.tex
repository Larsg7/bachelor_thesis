Computing masses of the particles in the Standard Model and investigating the force between them is subject of many thesis and papers and will be the topic of many to come. This work focuses on particles interacting via the strong force and hence utilizes the theory of QCD. The elementary particles of QCD are called quarks which come in six flavors. Gluons on the other hand exchange the strong force. Gluons are massless bosons and carry a color charge similar to the quarks.

Hadrons are particles consisting of quarks and one distinguishes between mesons, hadrons composed of a quark-antiquark pair, and baryons which are bounded states of three quarks. Physical properties of hadrons arise from the quantum numbers of their constituents. These properties such as mass and electrical charge can be experimentally tested. It is therefore crucial for an elementary theory like QCD to reproduce these measurements to certain precision. Computing physical properties of QCD can usually only achieved by a numerical treatment.

One theory to treat QCD effectively on computers is called \textit{lattice QCD}. In \textit{lattice QCD} one defines a finite space-time lattice composed of a series of discrete points. On such lattices the masses of mesonic states can be obtained from so called correlators which are vacuum expectation values of hadron creation operators. These correlators can be numerically computed and the mass is extracted by an exponential fit.

When finding suitable creation operators in such a way that they model the right quantum numbers one has some degrees of freedom constructing them. This freedom is used by a method called \textit{distillation} which describes a way to construct hadron creation numbers using eigenvectors of the lattice laplacian.

This thesis describes a first test implementing the computation of meson correlation function using distillation in Marc Wagner's work group and the results this implementation yielded.