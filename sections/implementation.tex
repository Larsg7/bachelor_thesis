\subsection{Overview}
    The computation of the correlation function on one gauge configurations can be broken down into the following steps
    \begin{enumerate}
        \item computation of $N$ eigenvectors
        \item generation of $T \times 4 \times N$ sources
        \item computation of as many inversions as there are sources
        \item calculation of the perambulator and distilled gamma matrix
        \item computation of the correlation function
    \end{enumerate}
    This work builds on the contraction code written by people of Marc Wagner's group, code by Jan Kruse and the aforementioned \verb+tmLQCD+ package. For documentation about steps 1 through 3 please see \cite{bachelor_thesis_jan}.



\subsection{Modules}
    The following modules written in C++11 were added to the existing contraction code in \verb+.../distillation/+. Every module consists of a \verb+main.cpp+ and a \verb+module.h+ and \verb+module.cpp+ file. The documentation of each function can be found in the corresponding .h file.  
    \begin{itemize}
        \item \verb+perambulator/+: Calculates the perambulator using the sources and inverted sources.
        \item \verb+distilled_gamma/+: Computes the distilled gamma matrix given the calculated eigenvectors.
        \item \verb+correlator_trace/+: Computes the correlator for a given perambulator and distilled gamma matrix.
    \end{itemize}
    
    In addition to the above a \verb+MultiVector+ class has been written which can store a matrix of arbitrary dimension. Such a matrix can then be written to the disk. The definition of this class can be found in \verb+helper/multivector.hpp+. It is used to store the perambulator, distilled gamma matrix and correlator trace.
    
    For a more technical documentation see the provided \verb+README+ files and comments inside the source code.
    
    \subsubsection{Computation of the perambulators}
        This program computes the perambulator for all sources and inverted sources in two directories.
        The sources and the inverted sources \emph{have to be} in the format
        \begin{verbatim}
source.CONF-ID.time##.vec##.spin#
source.CONF-ID.time##.vec##.spin#.CONF-ID.00.inverted\end{verbatim}
        respectively. Here \verb+#+ stands for a digit and \verb+CONF-ID+ is usually a 4-digit number. The program will read in all sources and inverted sources for one pair $(k,k')$ into memory at once. Hence running the program needs for the gauge configurations used about 48GB of RAM.\\The program can be started by calling
        \begin{verbatim}
./perambulator path-to-sources path-to-inverted-sources conf_id
               path-to-target-dir num_vec T L\end{verbatim}
        where \verb+num_vec+ is the total number of eigenvectors $N$, \verb+T+ is the time dimension and \verb+L+ the spacial dimension of the lattice. Both \verb+path-to-sources+ and \verb+path-to-inverted-sources+ are paths to the directory where the sources and inverted sources live. The perambulator will be saved to the directory given by \verb+path-to-target-dir+ in the format
        \begin{verbatim}
perambulator.CONF-ID.num_vec##.time##\end{verbatim}
        using the \verb+MultiVector+ storage type mentioned earlier.
    
    \subsubsection{Computation of the distilled gamma matrix}
        This program computes the distilled gamma matrix for a given eigenvector. The gamma matrix $\Gamma$ can be changed inside the source code. The distilled gamma matrix will be saved in the format
        \begin{verbatim}
gamma.CONF-ID.num_vec##.time##\end{verbatim}
        again using the \verb+MultiVector+ storage type.\\
        The program can be started by calling
        \begin{verbatim}
./gamma path-to-ev conf_id path-to-target-dir num-vec T L\end{verbatim}
        \verb+path-to-ev+ is the path to the file which holds the eigenvector.
        
    \subsubsection{Computation of the correlation function}
        To calculate the correlation function call
        \begin{verbatim}
./trace path-to-gamma-matrix path-to-perambulator conf_id 
        path-to-target-dir num_vec T\end{verbatim}
        Here \verb+path-to-gamma-matrix+ and \verb+path-to-perambulator+ are the paths to the files holding the distilled gamma matrix and perambulator calculated earlier. The correlator will be saved in the format
        \begin{verbatim}
trace.CONF-ID.num_vec##.time##\end{verbatim}
        This file will contain all $T^2$ elements of the correlation function. To read the trace and calculate the mean values for every $\Delta t$ run
        \begin{verbatim}
./trace -r path-to-trace T\end{verbatim}
        This will print all values in the format $\Delta t$ $real(C)$ $imag(C)$ where $C$ is the correlation function. The output of this operation is the basis for the results in section \ref{section:results}.
        
    \subsubsection{Python control script}
        In addition to the modules already mentioned a python script to control the simulation was written. It can find all files necessary for the simulation and can set up job scripts automatically. The script can be found inside the folder \verb+.../distillation/scripts/+ alongside its documentation. To start the script run
        \begin{verbatim}
./run.py\end{verbatim}
        The script will walk you through the setup process. It is written in python version 3.6.
    