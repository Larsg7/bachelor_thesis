\documentclass{style}

\begin{document}
    \section{Introduction}
    
	\section{Theoretical basics}
	\subsection{Conventions}
	Throughout this work natural units will be used, so that
	$$\hbar = c = a = 1$$
	where $a$ is the distance between two lattice points.\\
	
	In any instance where one index appears twice the \textbf{Einstein summation notation} was used, an implicit summation is assumed.\\
	
	When working with path integrals the euclidean formulation of QCD is used where the real time $x_0$ is replaced with the imaginary time $x_0 = -ix_4$. Therefore the Minkowsky tensor $g_{\mu\nu}$ is replaced by $\delta_{\mu\nu}$:
	$$\delta_{\mu\nu} = 
	\begin{pmatrix}
	1 & 0 & 0 & 0 \\
	0 & 1 & 0 & 0 \\
	0 & 0 & 1 & 0 \\
	0 & 0 & 0 & 1 \\
	\end{pmatrix}
	$$\\
	\cite{Rothe2012}
	
	\subsection{QCD}
	Quantum chromodynamics is a SU(3) gauge Quantum field theory and describes quarks and gluons. The QCD lagrangian is defined as \cite{qcd1_script_philipsen}
	\begin{equation}
	    \Lagr_{QCD}(\psi,\bar{\psi},A_\mu) = \sum_f \bar{\psi}_f(x)(i\gamma_\mu\GaugeDerivative + m_f)\psi_f(x) + \frac{1}{4}F^a_{\mu\nu}F^a_{\mu\nu}
	    \label{qcd_lagrangian}
	\end{equation}
	$\psi(x)$ represents the quark field, $\GaugeDerivative = \partial_\mu - igA_\mu(x) = \partial_\mu-igT^aA^a_\mu(x)$ is the gauge covariant derivative and $F_{\mu\nu} = T^aF^a_{\mu\nu}$ is the field strength tensor of the theory. The latter is defined as
	\begin{equation}\label{strenth_tensor}
	\begin{aligned}
	    F_{\mu\nu} &= \partial_\mu A_\nu - \partial_\nu A_\mu-ig[A_\mu,A_\nu]\\
	    &= T^a(\partial_\mu A^a_\nu - \partial_\nu A^a_\mu + g f^{abc}A^b_\mu A^c_\nu)
	\end{aligned}
	\end{equation}
	
	% more about symmetries
	
	\subsection{Lattice-QCD}
	\subsection{Motivation}
	When calculating the gluon propagator in one loop perturbation theory \cite{qcd2_script_philipsen}
	\begin{equation}
	    \langle 0|T(A^a_\mu(x)A^b_\nu)|0\rangle = \frac{\delta}{i\delta J^{a\mu}} \frac{\delta}{i\delta J^{b\nu}} Z[J]\Big|_{J=0}
	\end{equation}
	where $Z[J]$ is the generating functional and $J$ the source terms of the gluons, we find for the gluon propagator $D_{\mu\nu}$
	\begin{equation}
	    i D_{\mu\nu}(k) = i D_{F,\mu\nu} + i D_{F,\mu\alpha} i\Pi_{\alpha\beta} i D_{F,\beta\nu} + ...
	\end{equation}
	$D_{F,\mu\nu}$ is the feynman propagator and $\Pi_{\mu\nu}$ is the gluon self energy. The former is defined as
	\begin{equation}
	    iD_{F,\mu\nu}(k) = -
	\end{equation}
	
	% motivation: divergent integral -> regularisation
	
	\subsection{Mesons}
	
	\newpage
	
	\section{Distillation}
	% descibe distillation
	The term distillation describes a quark-field smearing algorithm to compute hadron correlation functions. It was first described in 2009 by Michael Peardon \cite{distillation_paper}. Distillation promisises efficient computations of all-to-all quark propagators.\\
	
	Smearing is used to apply a smoothing function to the field before applying the creation operators. Smearing should remove short-range modes while keeping as many symmetries as possible. Distillation makes use of a gauge-covariant quark smearing based on the lattice Laplacian which is defined as follows
	\begin{equation}
	    -\nabla^2_{xy} = 6\delta_{xy} - \sum^3_{j=1}(\tilde{U}_j(x,t)\delta_{x+j,y} + \tilde{U}^\dagger_j(x-y,t)\delta_{x-j,y})
	\end{equation}
	
	Starting form the lattice Laplacian one can define a simple smearing operator
	\begin{equation}
	    J_{\omega,n}(t) = (1+\frac{\omega\nabla^2(t)}{n})^n
	\end{equation}
	
	$\omega$ and $n$ are tunable parameters. For large $n$, $J$ defines the exponential of $\omega\nabla^2$ which suppresses higher eigenmodes of the lapplacian hence only a few of the lowest modes contribute to J. % WHY?
	\begin{equation}
        \lim_{n\rightarrow\infty} J_{\omega,n}(t) = \exp(\omega\nabla^2)
	\end{equation}
	
	% decribe the different calculations in program
	
	\section{Implementation}
	
	\section{Results}
	
	\section{Discussion}
	
	\bibliography{ref}
	

\end{document}